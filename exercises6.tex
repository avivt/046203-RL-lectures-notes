\section{Exercises}

\begin{exercise}[\textbf{The $c \mu$ rule revisited}]
Consider again the job processing domain of Exercise \ref{ex:c_mu}:

$N$ jobs are scheduled to run on a single server. At each time step $(t=0,1,2,�)$, the sever may choose one of the remaining unfinished jobs to process. If job $i$ is chosen, then with probability ${\mu _i} > 0$ it will be completed, and removed from the system; otherwise the job stays in the system, and remains in an unfinished state. Notice that the job service is memoryless - the probability of a job completion is independent of the number of times it has been chosen.
Each job is associated with a waiting cost ${c_i} > 0$ that is paid for each time step that the job is still in the system. The server's goal is minimizing the total cost until all jobs leave the system.

This time, we will solve the problem numerically, testing the various dynamic programming and reinforcement learning algorithms we have learned so far.

We consider the following specific case, with $N = 5$:

\begin{center}
\begin{tabular}{|l|l|l|l|l|l|}
\hline
{\bf $i$}  & {\bf 1} & {\bf 2} & {\bf 3} & {\bf 4} & {\bf 5} \\ \hline
${\mu _i}$ & 0.6     & 0.5     & 0.3     & 0.7     & 0.1     \\ \hline
${c_i}$    & 1       & 4       & 6       & 2       & 9       \\ \hline
\end{tabular}
\end{center}

\paragraph{Part 1 - Planning}
In this part all the Matlab functions may use the true model (i.e., the ${\mu _i}$'s and ${c_i}$'s).
\begin{enumerate}
  \item How many states and actions are in the problem?
\end{enumerate}  
  Let $S$ denote the state space. A deterministic policy $\pi $ is a mapping from states to actions. In our case, it will be represented in Matlab as a vector of length $|S|$, in which each element denotes the selected action ($1, \ldots ,N$) for the corresponding state.
\begin{enumerate} 
\setcounter{enumi}{1} 
  \item Write a function (in Matlab) that take as input a policy, and returns the corresponding value function ${V^\pi }$, also represented as a vector of length $|S|$. You can solve it either directly by matrix inversion, or iteratively. Remember that the value of the terminal state (no more jobs left) is zero by definition.
  \item Plot the values of the policy ${\pi _c}$ that selects the job with the maximal cost ${c_i}$, from the remaining unfinished jobs.
  \item Write a function that calculates the optimal policy ${\pi ^*}$ using the policy iteration algorithm, and execute it, starting from the initial policy ${\pi _c}$. For each iteration of the algorithm, plot the value of the initial state ${s_0}$ (no jobs completed) for the current policy. How many steps are required for convergence?
  \item Compare the optimal policy ${\pi ^*}$ obtained using policy iteration to the $c\mu $ law. Also plot ${V^{{\pi ^*}}}$ vs. ${V^{{\pi _c}}}$.
  \item Write a simulator of the problem: a function that takes in a state $s$ and action $a$, and returns the cost of the state $c(s)$, and a random next state $s'$, distributed according to the transition probabilities of the problem.
\end{enumerate}

\paragraph{Part 2 - Learning}
In this part the learning algorithms cannot use the true model parameters, but only have access to the simulator function written above.

\begin{enumerate}
\setcounter{enumi}{6}
  \item \textbf{Policy evaluation:} consider again the policy ${\pi _c}$, and use the TD(0) algorithm to learn the value function ${V^{{\pi _c}}}$. Start from ${\hat V_{TD}}(s) = 0$ for all states. Experiment with several step size ${\alpha _n}$ schedules:
      \begin{enumerate}
        \item ${a_n} = 1/\left( {no.\,of\,visits\,to\,{s_n}} \right)$
        \item ${a_n} = 0.01$
        \item ${a_n} = \frac{{10}}{{100 + \left( {no.\,of\,visits\,to\,{s_n}} \right)}}$
      \end{enumerate}
For each step-size schedule, plot the errors ${\left\| {{V^{{\pi _c}}} - {{\hat V}_{TD}}} \right\|_\infty }$ and $\left| {{V^{{\pi _c}}}({s_0}) - {{\hat V}_{TD}}({s_0})} \right|$ as a function of iteration $n$. Explain the motivation behind each step-size schedule, and how it reflects in the results.

  \item Now, run policy evaluation using TD($\lambda $). Choose your favorite step-size schedule, and plot the errors ${\left\| {{V^{{\pi _c}}} - {{\hat V}_{TD}}} \right\|_\infty }$ and $\left| {{V^{{\pi _c}}}({s_0}) - {{\hat V}_{TD}}({s_0})} \right|$ as a function of iteration $n$ for several choices of $\lambda $. Repeat each experiment $20$ times and display average results.
  \item \textbf{Q-learning:} run Q-learning to find the optimal policy. Start from $\hat Q(s,a) = 0$ for all states and actions. Experiment with the 3 previous step-size schedules. Use $\epsilon$-greedy exploration, with $\epsilon = 0.1$.
      
      For some $\hat Q$, let  ${\pi _{\hat Q}}$ denote the greedy policy w.r.t. $\hat Q$, i.e., ${\pi _{\hat Q}}(s) = \arg {\min _a}\hat Q(s,a)$.
      
For each step-size schedule, plot the errors ${\left\| {{V^*} - {V^{{\pi _{\hat Q}}}}} \right\|_\infty }$ and $\left| {{V^*}({s_0}) - \arg {{\min }_a}\hat Q({s_0},a)} \right|$ as a function of iteration $n$. You may use the policy evaluation function you wrote in (2) to calculate ${V^{{\pi _{\hat Q}}}}$. If this step takes too long, you may plot the errors only for $n = 100,200,300,...$ etc.

  \item Repeat the previous Q-learning experiment for your favorite step-size schedule but now with $\epsilon = 0.01$. Compare.
\end{enumerate}
\end{exercise}

